\documentclass[aspectratio=169,xcolor=dvipsnames,12pt]{beamer}
\usetheme{Candy}

\usepackage[english]{babel}
\usepackage{hyperref}
\usepackage{graphicx}
\usepackage{booktabs}
\usepackage{euler}
\usepackage{pifont}
\usepackage{fontawesome}
\usepackage{academicons}
\usepackage{tikz}
\usepackage{tikz-cd}

\title[short title]{Candy Beamer Theme}
\subtitle{Subtitle}

\author[LuMe]{Luciano Melodia}

\institute[FAU]
{
    Department of Mathematics \\
    Friedrich-Alexander Universität Erlangen-Nürnberg
}
\date{\today}

\begin{document}

\begin{frame}
    \titlepage
\end{frame}

\begin{frame}{Overview}
    \tableofcontents
\end{frame}

\section{First Section}

\begin{frame}{Bullet Points}
    \begin{itemize}
        \item Lorem ipsum dolor sit amet, consectetur adipiscing elit
        \item Aliquam blandit faucibus nisi, sit amet dapibus enim tempus eu
        \item Nulla commodo, erat quis gravida posuere, elit lacus lobortis est, quis porttitor odio mauris at libero
        \item Nam cursus est eget velit posuere pellentesque
        \item Vestibulum faucibus velit a augue condimentum quis convallis nulla gravida
    \end{itemize}
\end{frame}


\begin{frame}{Blocks of Highlighted Text}
    In this slide, some important text will be \alert{highlighted} because it's important. Please, don't abuse it.

    \begin{block}{Block}
        Sample text in Apricot box.
    \end{block}

    \begin{alertblock}{Alertblock}
        Sample text in Melon box.
    \end{alertblock}

    \begin{examples}
        Sample text in Peach box. The title of the block is ``Examples".
    \end{examples}
\end{frame}

%------------------------------------------------

\begin{frame}{Multiple Columns}
    \begin{columns}[c]
        \column{.45\textwidth}
        \textbf{Heading}
        \begin{enumerate}
            \item Statement
            \item Explanation
            \item Example
        \end{enumerate}

        \column{.5\textwidth}
        Lorem ipsum dolor sit amet, consectetur adipiscing elit. Integer lectus nisl, ultricies in feugiat rutrum, porttitor sit amet augue. Aliquam ut tortor mauris. Sed volutpat ante purus, quis accumsan dolor.
    \end{columns}
\end{frame}

\section{Second Section}

\begin{frame}{Table}
    \begin{table}
        \begin{tabular}{l l l}
            \toprule
            \textbf{Treatments} & \textbf{Response 1} & \textbf{Response 2} \\
            \midrule
            Treatment 1         & 0.0003262           & 0.562               \\
            Treatment 2         & 0.0015681           & 0.910               \\
            Treatment 3         & 0.0009271           & 0.296               \\
            \bottomrule
        \end{tabular}
        \caption{Table caption}
    \end{table}
\end{frame}

\begin{frame}{Theorem}
    \begin{theorem}[Fun with series]
        $\sin x = \sum\limits_{n = 1}^\infty  {\frac{{\left( { - 1} \right)^{n - 1} x^{2n - 1} }}{{\left( {2n - 1} \right)!}}}$
    \end{theorem}
\end{frame}

%------------------------------------------------

\begin{frame}{Figure}
    \begin{figure}
      \newcommand{\inputs}{5}
\newcommand{\hiddens}{3}
\newcommand{\outputs}{5}

\begin{tikzpicture}
\foreach \i in {1,...,\inputs}
{
	\node[circle,
		minimum size = 6mm,
		fill=Apricot] (Input-\i) at (0,-\i) {};
}

\foreach \i in {1,...,\hiddens}
{
	\node[circle,
		minimum size = 6mm,
		fill=teal!50,
		yshift=(\hiddens-\inputs)*5 mm
	] (Hidden-\i) at (2.5,-\i) {};
}

\foreach \i in {1,...,\outputs}
{
	\node[circle,
		minimum size = 6mm,
		fill=purple!50,
		yshift=(\outputs-\inputs)*5 mm
	] (Output-\i) at (5,-\i) {};
}

\foreach \i in {1,...,\inputs}
{
	\foreach \j in {1,...,\hiddens}
	{
		\draw[-latex, shorten >=1pt] (Input-\i) -- (Hidden-\j);
	}
}

\foreach \i in {1,...,\hiddens}
{
	\foreach \j in {1,...,\outputs}
	{
		\draw[-latex, shorten >=1pt] (Hidden-\i) -- (Output-\j);
	}
}

\foreach \i in {1,...,\inputs}
{
	\draw[latex-, shorten >=1pt] (Input-\i) -- ++(-1,0)
		node[left]{$x_{\i}$};
}

\foreach \i in {1,...,\outputs}
{
	\draw[-latex, shorten >=1pt] (Output-\i) -- ++(1,0)
		node[right]{$\varphi(x_{\i})$};
}

\end{tikzpicture}

      \caption{Wow, just three hidden neurons!}
    \end{figure}
\end{frame}

%------------------------------------------------

\begin{frame}[fragile]
    \frametitle{Citation}
    An example of the \verb|\cite| command to cite within the presentation:\\~

    This statement requires citation \cite{p1}.\footnote{We inserted even an aligned footnote, that looks just awesome!}
\end{frame}


\begin{frame}{References}
    \footnotesize{
        \begin{thebibliography}{99}
          \bibitem[Melodia, 2021]{p1} Luciano Melodia, Richard Lenz (2021)
          \newblock Homological Time Series Analysis of Sensor Signals from Power Plants
          \newblock \emph{Machine Learning for Irregular Time Series}, 45 -- 62.
          \bibitem[Melodia, 2021]{p1} Luciano Melodia, Richard Lenz (2021)
          \newblock Estimate of the Neural Network Dimension Using Algebraic Topology and Lie Theory
          \newblock \emph{Image Mining: Theory and Applications} (12665), 15 -- 29.
          \bibitem[Melodia, 2021]{p1} Luciano Melodia, Richard Lenz (2020)
          \newblock Persistent Homology as Stopping-Criterion for Voronoi Interpolation
          \newblock \emph{International Workshop on Combinatorial Image Analysis} (12148), 29 -- 44.
        \end{thebibliography}
    }
\end{frame}


\begin{frame}
    \Huge{\centerline{\textbf{The End}}}
\end{frame}


\end{document}
